
%%%%%%%%%%%%%%%%%%%%%%%%%%%%%%%%%%%%%%%%%%%%%%%%%%%%%%%%%%%%%%%%%%%%%%%%%
%           Capítulo 2: MARCO TEÓRICO - REVISIÓN DE LITERATURA
%%%%%%%%%%%%%%%%%%%%%%%%%%%%%%%%%%%%%%%%%%%%%%%%%%%%%%%%%%%%%%%%%%%%%%%%%

\chapter{Marco teórico}

En este capítulo, normalmete se ponen todas las ecuaciones que se van a usar en la tesis, así ya nomás se hace rferencia a la ecuación tal o "como se vió en el capítulo 2", y esas cosas. 

\chapter{ Definiciones }

    \section{ Black Scholes }

    \section{ Meta-algoritmos }
    
        \subsection{  Algoritmo evolutivo ( Meta algoritmo ) - Non Sorted Genetic Algorithm II }
    
    \section{ Algoritmos }
    
        \subsection{ Red neuronal con retropropagación }
    

\chapter{ El juego }

    Me he dado a la tarea de definir matemtaticamente los siguientes conceptos
    
    \section{ El juego (Reto actinver) }
        
        Las reglas del reto actinver.
    
    \section{ Bolsa } % Bloque regulador.
      
        \blindtext
        
   	\section{ Broker } % cobra comisiones y es muy estupido (por telefono) por internet es lo mismo  
     
    \section{ Mercado } 
    
        \blindtext
    
    \section{ Agente } % puede efectuar acciones a traves de una casa de bolsa o la bolsa en si
        
        \subsection{ Propiedades y metodos del agente }

                \blindtext
   
   	
   	\section{ Mundo real } % Aqui ocurren eventos que cambian los precios en el mercado
   	
   	\section{ Utilidad Horizonte Infinito }
    
    \section{ Instrumentos financieros: Acciones y otros... }
    
\chapter{ Bolsa mexicana de valores }

    En esta sección se definirá matemáticamente conceptos de la bolsa mexicana de valores, considerando las definiciones antes descritas. Tambien se dará una breve descripción para entender el sentido matematico.
    
    \section{ Bolsa mexicana de valores }
        
        La bolsa mexicana de valores es... blah blah blah... que tiene una definición matemática dada por mi... un conjunto de reglas, un historial, estados... $R$
       
    \section{ Mercado }
    
        \subsection{ Etapas del mercado }
        
        ¿Qué es un mercado?
        
        ¿Qué clases de sesiones existen?
        
        En las sesiones de remate se consideran las siguientes etapas operativas:...
       
        Horarios y etapas de la bolsa mexicana de valores.         
                
        \subsection{ Posturas en la bolsa mexicana de valores }   
        
            Aqui se explica que es una postura y sus atributos. 
            
                1. Vigencia 
                2. Volumen oculto
                3. Volumen minimo de ejecución
                4. Venta en corto
                
            Datos para ingreso y cancelación de posturas...
                
            \subsubsection{ Limitada }
            
            \subsubsection{ A mercado pura }
            
            \subsubsection{ Mercado con protecci\'on }
            
            \subsubsection{ Al cierre ( Etapa 1 y 2 ) }
            
            \subsubsection{ Mejor postura limitada }
            
            \subsubsection{ Precio medio }
            
            \subsubsection{ Precio Promedio del dia }
                
       \subsection{ Mercado }
        
        Medio que permite a compradores y vendedores de un bien o servicio.
        
            \subsubsection{ Segmentos del mercado }
            
            
        
        \subsection{ Opciones }
        
        Definition of 'Option'

        A financial derivative that represents a contract sold by one party (option writer) to another party (option holder). The contract offers the buyer the right, but not the obligation, to buy (call) or sell (put) a security or other financial asset at an agreed-upon price (the strike price) during a certain period of time or on a specific date (exercise date).

        Call options give the option to buy at certain price, so the buyer would want the stock to go up.

        Put options give the option to sell at a certain price, so the buyer would want the stock to go down.
        
        Investopedia explains 'Option'

        Options are extremely versatile securities that can be used in many different ways. Traders use options to speculate, which is a relatively risky practice, while hedgers use options to reduce the risk of holding an asset.

        In terms of speculation, option buyers and writers have conflicting views regarding the outlook on the performance of an underlying security.

        For example, because the option writer will need to provide the underlying shares in the event that the stock's market price will exceed the strike, an option writer that sells a call option believes that the underlying stock's price will drop relative to the option's strike price during the life of the option, as that is how he or she will reap maximum profit.

        This is exactly the opposite outlook of the option buyer. The buyer believes that the underlying stock will rise, because if this happens, the buyer will be able to acquire the stock for a lower price and then sell it for a profit.
        
            \subsubsection{ Call option }
        
                Opción de comprar a cierto precio
        
            \subsubsection{ Put option }
        
                Opción de vender a cierto precio.
        
            \subsubsection{ Go short }
        
                Acción de pedir prestado para comprar una acción con la intención de que suba de precio.
        
        \subsection{ Go long }
        
            Accion de comprar una acción con la intención de que suba su precio.
        
        \subsection{ Stop order }
        
            Es un threshold en el cual se vende la accion
            
        

\blindtext